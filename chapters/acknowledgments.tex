If there are two people to whom the beginning of this journey of PhD can be traced back to, they are Prof. Chandra R. Murthy of the Department of ECE, IISc, and Dr. K. G. Nagananda, a close colleague and a doting mentor of mine. While Prof. Chandra introduced me to the world of academic research and sowed in me the seeds that would later germinate and embrace the vast ocean that research is, Nanda taught me the art of searching for pearls hidden deep beneath this ocean. While one taught me how to fly long distances looking for a flower, the other taught me the art of sucking the ambrosia from the flower without robbing the flower of its form and fragrance. If not for these noble souls, I would not have had the courage to pursue PhD. I am immensely thankful to both of them.

In my search for pearls in the ocean of research, I found one in my adviser, Prof. Rajesh Sundaresan. From heartily welcoming me into his lab space to making me feel at home, from patiently making me understand my paltry background in mathematics to guiding me gracefully towards speaking fluently in a language that he naturally spoke, from giving me the space and time to do things at my own pace to encouraging me to do things in my own way, from being an astute adviser who recognises the many infirmities of his students to thinking many steps ahead towards turning them into strengths, Prof. Rajesh has been with me through the thick and thin of my PhD journey. In his absence, he has reminded me that he is the sturdy earth beneath my feet along this journey, always present and ever ready to take on weight. For all the important lessons in research and in life that Prof. Rajesh has taught me, I shall remain forever indebted to him. I hope to brighten the lives of many students in much the same way that Prof. Rajesh did of mine, perhaps the least I can do to carry forward his legacy.

I am thankful to Prof. Utpal Mukherji, Prof. Navin Kashyap, Prof. Himanshu Tyagi, and Prof. Parimal Parag for serving as lighthouses along the journey. Their words of encouragement during every step of my PhD have been invaluable. From providing me with multiple opportunities to work alongside them in various capacities, be as a teaching assistant or otherwise, to readily agreeing to serve as my professional referees, they have been a true source of support. In particular, I would like to thank Prof. Navin and Prof. Himanshu for serving on the annual progress review committee and providing a platform for me to share with them the difficulties I faced along the journey without the fear of being judged. 

I am immensely thankful to all the professors of the Department of ECE and the Department of Mathematics at IISc for the superior quality lectures that I was fortunate to listen to. I am also thankful to the professors and students of TIFR Mumbai, IIT Bombay, IIT Madras, ISI Bangalore, ISI Delhi, and ISI Kolkata for the opportunities given to me to attend many of their seminars, workshops and lectures, and present my work at these places on several occasions.

An integral part of my PhD journey, one that will stay the closest to my heart, is the people with whom I associated myself while at IISc. Kishan, Chatan, Nihesh, Sarath, Krishna, Prakash, Hemanth, Akhil, Thiru, Nidhin, Bharath, Surabhi, Vinay, Garima, Chinmaya, Hari, Lakshmi Priya, Prathamesh, Sahasranand, Lekshmi, Bala, Sarvendranath, Vaishali, Rooji, Tarun -- these names will stay etched in my heart for the rest of my life. There are many more names that I have not mentioned here, courtesy paucity of space. 

This work was supported in part by the Science and Engineering Research Board (SERB) of the Department of Science and Technology, in part by the Ministry of Human Resource Development, Government of India, in part by the Centre for Networked Intelligence (CNI), IISc, and in part by the Robert Bosch Center for Cyber-Physical Systems (RBCCPS), IISc. I am thankful to RBCCPS for providing me the monetary support to travel to France to present my work at the 2019 IEEE International Symposium on Information Theory (ISIT). I would like to thank Lalitha Bai and Mr. Srinivasamurthy for extending support to me in matters concerning travel and disbursement of salary on a regular basis. 

This journey could not have been completed without the blessings of my parents and my gurus. I am thankful to Providence for gifting me a loving sister who cared for our parents in my absence. I am indebted to the many relatives whose love, care and words of encouragement kept me going along the journey. I only regret not being able to give them a satisfactory reply every time they asked me how much longer it would take me to finish my PhD studies. I am glad that they will not have to ask me anymore! 









